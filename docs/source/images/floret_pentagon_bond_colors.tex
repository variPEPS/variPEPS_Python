\documentclass[tikz]{standalone}

\usepackage{amsfonts}
\usepackage{amsmath}
\usepackage{braket}

\usepackage{tikz}
\usetikzlibrary{math, calc, decorations, positioning}

% load TikZ grafic definitions
%\input{gfx_TikZ}

% main document
\begin{document}

\pgfdeclarelayer{v1}    % background  layer
\pgfdeclarelayer{v2}    % foreground  layer
\pgfsetlayers{v1,main,v2}  % set the order of the layers (main is the standard layer)
\tikzset{
  bonda/.style={black, line width=.65pt},
  bondb/.style={blue!80!white, line width=.65pt},
  bondc/.style={green!80!black, line width=0.65pt}}

\begin{tikzpicture}
  \tikzmath {
    \ang0 = 0;
    \lb = 0.5;
    \la = 2*\lb;
    int \sa;
    int \sb;
    \nt = 1;
    for \sa  in {-\nt,...,\nt}{
      int \sbmin, \sbmax;
      \sbmin = max(-\nt, -\nt-\sa);
      \sbmax = min(\nt, \nt-\sa);
      for \sb  in {\sbmin,...,\sbmax}{
        coordinate \oo;
        \oo = (70.89339464913091+\ang0:9.16515138991168*\sa*\lb/2)
        + (10.893394649130906+\ang0:9.16515138991168*\sb*\lb/2);
        int \i;
        for \i in {0,1,2,3,4,5}{
          \angr = 60*\i+\ang0;
          \a1 = 0+\angr;
          \a2 = 60+\angr;
          \a3 = 120+\angr;
          \a4 = 180+\angr;
          {
            \draw[bonda] (\oo) coordinate (oo) coordinate (oo a\sa b\sb) --
            ++(\a1:\la) coordinate (pa \i) coordinate (pa \i a\sa b\sb)
            ++ (\a2:\lb) coordinate (pb \i) coordinate (pb \i a\sa b\sb)
            ++ (\a3:\lb) coordinate (pc \i) coordinate (pc \i a\sa b\sb)
            ++ (\a4:\lb) coordinate (pd \i) coordinate (pd \i a\sa b\sb);
          };
        };
        {

          % \draw (pa 0) -- (pb 0) -- (pc 0) -- (pd 0);
          % \draw (pa 1) -- (pb 1) -- (pc 1) -- (pd 1);
          % \draw (pa 2) -- (pb 2) -- (pc 2);
          % \draw (pc 5) -- (pa 0);
          \draw[bondb]  (pa 0) -- (pb 0);
          \draw[bondc] (pb 0) -- (pc 0);
          \draw[bondc] (pc 0) -- (pd 0);
          \draw[bondb] (pa 1) -- (pb 1);
          \draw[bondc] (pb 1) -- (pc 1);
          \draw[bondc] (pc 1) -- (pd 1);
          \draw[bondb] (pa 2) -- (pb 2);
          \draw[bondc] (pb 2) -- (pc 2);
          \draw[bondc] (pc 5) -- (pa 0);
        };
        if (\sb==\sbmin) then {
          {
            \draw (pc 2)[bondc] -- (pd 2);
            \draw (pd 2)[bondb] -- (pb 3);
            \draw (pb 3)[bondc] -- (pc 3);
          };
        };
        if \sa<1 then {
          if (\sb==\sbmin || \sa==-\nt then {
            {
              \draw[bondc] (pc 3) -- (pd 3);
              \draw[bondb] (pd 3) -- (pb 4);
              \draw[bondc] (pb 4) -- (pc 4);
            };
          };
          if (\sb==\sbmax || \sa==-\nt then {
            {
              \draw[bondc] (pc 4) -- (pd 4);
              \draw[bondb] (pd 4) -- (pb 5);
              \draw[bondc] (pb 5) -- (pc 5);
            };
          };
        };
        {
          \draw[fill=red] (oo) circle [radius=1.5pt];
          \draw[fill=red] (pa 0) circle [radius=1.5pt];
          \draw[fill=red] (pa 1) circle [radius=1.5pt];
          \draw[fill=red] (pa 2) circle [radius=1.5pt];
          \draw[fill=red] (pa 3) circle [radius=1.5pt];
          \draw[fill=red] (pa 4) circle [radius=1.5pt];
          \draw[fill=red] (pa 5) circle [radius=1.5pt];

          \draw[fill=red] (pb 0) circle [radius=1.5pt];
          \draw[fill=red] (pb 1) circle [radius=1.5pt];
          \draw[fill=red] (pb 2) circle [radius=1.5pt];
          \draw[fill=red] (pb 3) circle [radius=1.5pt];
          \draw[fill=red] (pb 4) circle [radius=1.5pt];
          \draw[fill=red] (pb 5) circle [radius=1.5pt];

          \draw[fill=red] (pc 0) circle [radius=1.5pt];
          \draw[fill=red] (pc 1) circle [radius=1.5pt];
          \draw[fill=red] (pc 2) circle [radius=1.5pt];
          \draw[fill=red] (pc 3) circle [radius=1.5pt];
          \draw[fill=red] (pc 4) circle [radius=1.5pt];
          \draw[fill=red] (pc 5) circle [radius=1.5pt];
        };
      };
    };
    {
      \tikzset{ucirc/.style={fill=blue}}
      \begin{pgfonlayer}{v2}
        % \draw[line width=1pt,-stealth] (oo a0b0) -> (oo a0b1);
        % \draw[line width=1pt,-stealth] (oo a0b0) -> (oo a1b0);
        % \draw[line width=1pt,dashed,gray] (oo a0b1) -> (oo a1b1);
        % \draw[line width=1pt,dashed,gray] (oo a1b0) -> (oo a1b1);
      \end{pgfonlayer}{v2}
      % \draw[ucirc] (oo a0b0) circle [radius=1.5pt];
      % \draw[ucirc] (pb 0a0b0) circle [radius=1.5pt];
      % \draw[ucirc] (pc 0a0b0) circle [radius=1.5pt];
      % \draw[ucirc] (pd 0a0b0) circle [radius=1.5pt];

      % \draw[ucirc] (pb 1a0b1) circle [radius=1.5pt];
      % \draw[ucirc] (pc 1a0b1) circle [radius=1.5pt];
      % \draw[ucirc] (pd 1a0b1) circle [radius=1.5pt];

      % \draw[ucirc] (pa 5a1b0) circle [radius=1.5pt];
      % \draw[ucirc] (pd 5a1b0) circle [radius=1.5pt];
    };
    {
      % \node at (3,-1) {$v_1 = \frac{a}{2}\mqty(3\\5\sqrt{3})$};
      % \node at (6,-1) {$v_2 = \frac{a}{2}\mqty(9\\\sqrt{3})$};
    };
  }
\end{tikzpicture}

\end{document}

%%% Local Variables:
%%% mode: latex
%%% TeX-master: t
%%% End:
